%%%%%%%%%%%%%%%%% DO NOT CHANGE HERE %%%%%%%%%%%%%%%%%%%% 
%%%%%%%%%%%%%%%%%%%%%%%%%%%%%%%%%%%%%%%%%%%%%%%%%%%%%%%%%%{
    \documentclass[twoside,11pt]{article}
    %%%%% PACKAGES %%%%%%
    \usepackage{pgm2016}
    \usepackage{amsmath}
    \usepackage{algorithm}
    \usepackage[noend]{algpseudocode}
    \usepackage{subcaption}
    \usepackage[english]{babel}	
    \usepackage{paralist}	
    \usepackage[lowtilde]{url}
    \usepackage{fixltx2e}
    \usepackage{listings}
    \usepackage{color}
    \usepackage{hyperref}
    \usepackage{natbib}
    \usepackage{auto-pst-pdf}
    \usepackage{pst-all}
    \usepackage{pstricks-add}
    \usepackage{float} % for [H] placement

    %%%%% MACROS %%%%%%
    \algrenewcommand\Return{\State \algorithmicreturn{} }
    \algnewcommand{\LineComment}[1]{\State \(\triangleright\) #1}
    \renewcommand{\thesubfigure}{\roman{subfigure}}
    \definecolor{codegreen}{rgb}{0,0.6,0}
    \definecolor{codegray}{rgb}{0.5,0.5,0.5}
    \definecolor{codepurple}{rgb}{0.58,0,0.82}
    \definecolor{backcolour}{rgb}{0.95,0.95,0.92}
    \lstdefinestyle{mystyle}{
       backgroundcolor=\color{backcolour},  
       commentstyle=\color{codegreen},
       keywordstyle=\color{magenta},
       numberstyle=\tiny\color{codegray},
       stringstyle=\color{codepurple},
       basicstyle=\footnotesize,
       breakatwhitespace=false,        
       breaklines=true,                
       captionpos=b,                    
       keepspaces=true,                
       numbers=left,                    
       numbersep=5pt,                  
       showspaces=false,                
       showstringspaces=false,
       showtabs=false,                  
       tabsize=2
    }
    \lstset{style=mystyle}
%%%%%%%%%%%%%%%%%%%%%%%%%%%%%%%%%%%%%%%%%%%%%%%%%%%%%%%%%% 
%%%%%%%%%%%%%%%%%%%%%%%%%%%%%%%%%%%%%%%%%%%%%%%%%%%%%%%%%% }

%%%%%%%%%%%%%%%%%%%%%%%% CHANGE HERE %%%%%%%%%%%%%%%%%%%% 
%%%%%%%%%%%%%%%%%%%%%%%%%%%%%%%%%%%%%%%%%%%%%%%%%%%%%%%%%% {
\newcommand\course{xx25106}
\newcommand\courseName{CS Seminar}
\newcommand\semester{Fall 2025}
\newcommand\assignmentNumber{01}                             % <-- ASSIGNMENT #
\newcommand\studentName{Abdelrahman I. A. Abushbeka}                  % <-- YOUR NAME
\newcommand\studentEmail{s2520711@u.tsukuba.ac.jp}          % <-- YOUR NAME
\newcommand\studentNumber{202520711}                % <-- STUDENT ID #
%%%%%%%%%%%%%%%%%%%%%%%%%%%%%%%%%%%%%%%%%%%%%%%%%%%%%%%%%% }
%%%%%%%%%%%%%%%%%%%%%%%%%%%%%%%%%%%%%%%%%%%%%%%%%%%%%%%%%%

%%%%%%%%%%%%%%%%% DO NOT CHANGE HERE %%%%%%%%%%%%%%%%%%%% 
%%%%%%%%%%%%%%%%%%%%%%%%%%%%%%%%%%%%%%%%%%%%%%%%%%%%%%%%%%
%{

    \ShortHeadings{University of Tsukuba -  \course ~~ \courseName}{\studentName - \studentNumber}
    \firstpageno{1}
    
    \begin{document}
    
    \title{Difference Subspaces \& U-Net for Multi-spectral Change \& Damage Segmentation Using Sentinel-2 Satellite Multi-band Imaging}
    
    \author{\name \studentName \email \studentEmail \\
    \studentNumber
    \addr
    }
    
    \maketitle
%%%%%%%%%%%%%%%%%%%%%%%%%%%%%%%%%%%%%%%%%%%%%%%%%%%%%%%%%%

%%%%%%%%%%%%%%%%%%%%%%%%%%%%%%%%%%%%%%%%%%%%%%%%%%%%%%%%%% }
\begin{abstract}
\noindent
Accurate and timely assessment of disaster damage from satellite imagery is
crucial for emergency response and reconstruction. This work describes a
two--phase pipeline that combines classical subspace--based change detection
with supervised segmentation on multi--temporal Sentinel--2 data. In Phase~1
we implement Difference Subspace (DS) change detection on the Onera Satellite
Change Detection (OSCD) benchmark and on the unlabeled MultiSenGE dataset,
comparing DS against classical unsupervised baselines: pixel differencing,
change vector analysis (CVA), PCA--based difference (PCA--diff), Celik's
local PCA+K--means method, and IR--MAD. Phase~1 produces continuous change
maps and quantitative OSCD results (AUROC, F1, IoU) that serve as
interpretable priors. In Phase~2 we train U--Net and ResNet--U--Net models
for pixel--wise change segmentation on OSCD, using Sentinel--2 pre/post
bands alone and in combination with DS and PCA--diff change maps as
additional channels and fusion branches. On OSCD, raw Sentinel--2
segmentation achieves the best IoU/F1, while DS/PCA--diff priors slightly
improve AUROC and serve as valuable explanatory overlays. The code and
design are structured so that the same architecture and priors can be
transferred to damage--labeled datasets such as xBD, forming the basis of a
future Phase~3 for building damage segmentation.
\end{abstract}

\section{Introduction}

Remote sensing offers a global, repeatable view of urban and rural areas
before and after disasters. Robust damage assessment from satellite imagery
remains challenging: annotations are expensive, radiometric conditions vary,
and small changes in vegetation or illumination can confound simple
difference metrics. At the same time, classical linear subspace methods
provide a principled way to model spectral manifolds and their changes over
time \cite{FukuiMaki2015,FukuiSubspace}.

The long--term goal of this project is label--efficient \emph{disaster damage
segmentation} from multi--temporal satellite data. Instead of directly
training a deep model on a damage dataset, we decompose the problem into two
complementary phases:

\begin{itemize}
  \item \textbf{Phase~1:} Unsupervised Difference--Subspace (DS) change
  detection on Sentinel--2, evaluated on OSCD and visualized on MultiSenGE.
  The output is a set of continuous change maps and classical baselines that
  act as priors.
  \item \textbf{Phase~2:} Supervised change segmentation on OSCD, using
  Sentinel--2 pre/post bands and optionally DS/PCA--diff change maps as
  extra channels or fusion branches.
\end{itemize}

Both phases are designed with forward compatibility: by changing only the
dataset adapter and the number of output classes, the same pipeline can be
ported to damage--labeled datasets such as xBD \cite{Gupta2019xBD} in a
future Phase~3.

\section{Related Work}

\paragraph{Subspace methods and Difference Subspace.}
Subspace methods such as mutual subspace methods and Grassmannian learning
are widely used for image--set recognition and time--series modeling
\cite{FukuiSubspace}. Fukui and Maki introduced the \emph{Difference
Subspace} (DS) and its generalization for subspace--based methods
\cite{FukuiMaki2015}, providing a framework to separate common and
difference components between two subspaces. DS and higher--order variants
\cite{SecondOrderDS} have been applied to anomaly detection, showing that
difference subspaces can highlight subtle changes between signal subspaces.
Here we apply DS in the spectral domain of Sentinel--2 images.

\paragraph{Classical change detection.}
Unsupervised change detection in satellite images includes pixel
differencing, CVA, PCA--based methods, and multivariate approaches. PCA--diff
performs dimensionality reduction on the difference image and uses the
magnitude of projections onto leading principal components as change scores.
Celik proposed a local PCA + K--means detector that operates on patches of
the difference image \cite{Celik2009}, while IR--MAD iteratively reweights
canonical variates between pre/post images. We implement these as Phase~1
baselines.

\paragraph{Deep learning for change and damage.}
CNNs have been successfully applied to multi--temporal change detection
\cite{Daudt2018}. U--Net architectures \cite{Ronneberger2015} are standard
for semantic segmentation, and ResNets provide strong backbones
\cite{He2016ResNet}. For damage assessment, the xBD dataset introduced
large--scale pre/post imagery with building damage labels
\cite{Gupta2019xBD}. Our Phase~2 experiments deliberately use simple
U--Net/ResNet baselines on OSCD binary change masks to isolate the effect of
DS/PCA--diff priors rather than to compete directly with the latest
specialized change--detection architectures. The added value of our work is
to bring DS--style spectral priors and interpretability into a segmentation
pipeline that can later be attached to damage--labeled datasets.

\section{Datasets and Preprocessing}

\paragraph{OSCD.}
The Onera Satellite Change Detection (OSCD) dataset consists of 24
Sentinel--2 tiles with pre/post images (13 spectral bands) and binary change
masks at multiple resolutions \cite{Daudt2018}. We treat each city as a tile
with tensors \(X_{\text{pre}},X_{\text{post}} \in \mathbb{R}^{13 \times H
\times W}\) and a binary mask \(Y \in \{0,1\}^{1 \times H \times W}\). We
adopt the official split into 14 train and 10 test cities, and we choose a
small subset of train cities (e.g.\ \emph{abudhabi, aguasclaras, beihai})
for validation. For all tiles, we build a valid--pixel mask based on a
NODATA value and apply global per--band z--score normalization using
statistics computed over OSCD training data.

\paragraph{MultiSenGE.}
The MultiSenGE dataset provides about 8k unlabeled multi--temporal
Sentinel--2 patches (256\(\times\)256) over the Grand--Est region. In
Phase~1 we use the Sentinel--2 L2A bands as an unlabeled testbed: patches
are grouped by tile and spatial index, and date--sorted pairs (e.g.\ earliest
vs latest) are selected for DS visualization. Band statistics from OSCD are
reindexed to the MultiSenGE band order for consistent normalization.

\section{Phase 1: Difference--Subspace Change Detection}

For each OSCD tile we assemble normalized spectral matrices
\(X_1,X_2 \in \mathbb{R}^{d \times n}\) (pre/post, \(d=13\) bands). We fit
PCA bases \(\Phi,\Psi \in \mathbb{R}^{d \times r}\) with either fixed rank
or a variance threshold. We implement two DS variants following
\cite{FukuiMaki2015}: a residual--stacked construction using residual
projectors \(R_\Phi = I - \Phi\Phi^\top\), \(R_\Psi = I - \Psi\Psi^\top\),
and an eigen--based construction from \(G = \Phi\Phi^\top + \Psi\Psi^\top\).
On OSCD, both variants yield almost identical scores, so we use the
residual--stacked DS as default.

Given a pixel pair \((x_1,x_2)\) and DS basis \(D\), we use the projection
energy
\[
p_{\text{DS}} = \lVert D^\top (x_2 - x_1) \rVert_2^2
\]
as the main DS change score. We compare DS against pixel diff/CVA,
PCA--diff, Celik PCA+K--means \cite{Celik2009}, and IR--MAD. Scores are
normalized per tile and thresholded via unsupervised Otsu or a
train--calibrated global threshold. We evaluate AUROC on continuous scores
and IoU/F1 on binary masks.

On the OSCD test split, PCA--diff achieves the highest AUROC (\(\sim 0.81\)),
with F1/IoU slightly above DS and pixel diff. DS projection is competitive
(AUROC \(\sim 0.75\)) and often produces cleaner, less noisy change maps than
pixel differencing, while the cross--residual variant and IR--MAD are
weaker. Celik's method performs reasonably but is heavier computationally.
Qualitative OSCD figures show that DS/PCA--diff highlight many of the same
regions as the ground truth but also respond to subtle spectral changes that
are not labeled as change. On MultiSenGE, DS maps reveal diverse spectral
change patterns without labels.

The key outcome of Phase~1 is a set of interpretable, unsupervised change
maps (DS, PCA--diff, pixel diff, etc.) and a robust evaluation pipeline that
we reuse in Phase~2 as priors and analysis tools.

\section{Phase 2: OSCD Change Segmentation with DS/PCA Priors}

\paragraph{Models and inputs.}
Phase~2 introduces supervised change segmentation on OSCD, using the Phase~1
change maps as optional priors. The core baseline is a 2D U--Net
\cite{Ronneberger2015} with four encoder/decoder levels and a final
\(1\times 1\) convolution producing logits in \(\mathbb{R}^{1 \times H
\times W}\). We also implement a stronger U--Net with a ResNet--34 encoder
\cite{He2016ResNet}. Input feature configurations include: raw pre+post
bands (26 channels), raw+DS, raw+PCA--diff, raw+DS+PCA, and priors--only
(DS+PCA). In addition to naive concatenation, we design a small
\emph{PriorsFusionUNet} where raw bands and priors are first processed by
separate \(1\times 1\) convolutions before fusion.

\paragraph{Training and evaluation.}
We extract overlapping 256\(\times\)256 patches from OSCD tiles, apply random
flips and 90° rotations, and train with a combination of
BCEWithLogits and soft Dice loss on valid pixels. We use AdamW with a cosine
learning rate schedule over 50 epochs. We report mean IoU, mean F1, and
AUROC over the OSCD test split, and we generate combined figures showing
pre/post RGB, ground truth, DS/PCA--diff maps, prediction probabilities, and
binary segmentation masks.

\paragraph{Results.}
Raw Sentinel--2 segmentation is a strong baseline: a plain U--Net with raw
pre/post bands reaches mean IoU \(\approx 0.23\), mean F1 \(\approx 0.33\),
and AUROC \(\approx 0.87\). A ResNet--U--Net with raw bands achieves similar
IoU/F1. Adding DS or PCA--diff as extra channels does not improve IoU/F1 and
often slightly degrades them; combined DS+PCA priors behave similarly.
PriorsFusionUNet improves over naive concatenation in F1 and AUROC but still
does not surpass the best raw--only models in IoU/F1. In several
configurations AUROC increases modestly when priors are included, suggesting
that priors sharpen ranking but introduce more false positives at a fixed
threshold. Table~\ref{tab:oscd-seg-results} summarizes representative OSCD
test metrics (approximate) for key configurations.

\begin{table}[t]
  \centering
  \caption{Approximate OSCD test--split segmentation results for key
    configurations. Raw Sentinel--2 segmentation (U--Net/ResNet) achieves the
    best IoU/F1; priors improve AUROC slightly but tend to reduce IoU/F1.}
  \label{tab:oscd-seg-results}
  \begin{tabular}{lccc}
    \hline
    Model / input                    & mean IoU & mean F1 & AUROC \\
    \hline
    U--Net, raw pre+post             & 0.229 & 0.334 & 0.865 \\
    U--Net, raw + DS + PCA--diff     & 0.205 & 0.300 & 0.828 \\
    ResNet--U--Net, raw pre+post     & 0.230 & 0.336 & 0.826 \\
    ResNet--U--Net, raw + DS + PCA   & 0.206 & 0.305 & 0.845 \\
    PriorsFusionUNet, raw + DS + PCA & 0.207 & 0.311 & 0.854 \\
    \hline
  \end{tabular}
\end{table}

Per--tile visualizations confirm that DS/PCA--diff often highlight broader
regions of spectral change than the binary OSCD masks, including vegetation
and radiometric artefacts. Networks that consume priors tend to be more
sensitive to these regions, which is helpful for weak changes but harmful for
strict evaluation against OSCD labels.

\section{Discussion and Conclusion}

The two--phase pipeline provides a coherent story. Phase~1 shows that DS and
PCA--diff are strong, interpretable unsupervised change detectors on
Sentinel--2: PCA--diff has the best AUROC on OSCD among classical methods,
while DS is competitive and often cleaner than pixel diff. Phase~2
demonstrates that, on OSCD, a well--trained segmentation network on raw
Sentinel--2 data is hard to beat in IoU/F1 when evaluated strictly against
the provided change masks.

Even when priors do not improve IoU/F1, they remain useful in two ways.
First, they improve \textbf{explainability}: combined DS/PCA--diff and
segmentation figures reveal where the network follows or ignores unsupervised
spectral change signals, which matters when labels are noisy or incomplete.
Second, they offer \textbf{transferability}: the DS machinery is
dataset--agnostic and can be recomputed on future damage datasets such as xBD
\cite{Gupta2019xBD} and used as priors, pseudo--labels, or loss weights in a
Phase~3 damage--segmentation pipeline.

Future work includes longer training and multiple seeds, ImageNet
pretraining for the ResNet encoder, pseudo--label pretraining on MultiSenGE
using DS/PCA--diff targets, alternative uses of priors (e.g.\ loss weighting
on DS--strong regions), and extending from binary change masks (OSCD) to
multi--class damage levels on xBD/xBD--S12 via a dataset adapter and
multi--class output heads.

Overall, the project moves from classical difference subspaces and PCA--based
change detection to supervised change segmentation, with DS and PCA--diff
maps reused as priors and explanatory tools. The codebase and experimental
design are ready to be extended to true damage mapping in future phases.

\begin{thebibliography}{99}

\bibitem{FukuiMaki2015}
K.~Fukui and A.~Maki.
\newblock Difference subspace and its generalization for subspace-based
methods.
\newblock {\em IEEE Transactions on Pattern Analysis and Machine
Intelligence}, 37(11):2164--2177, 2015.

\bibitem{FukuiSubspace}
K.~Fukui.
\newblock Subspace methods.
\newblock Technical report, University of Tsukuba.
\newblock Overview/tutorial document on mutual subspace methods and related
approaches.

\bibitem{SecondOrderDS}
K.~Fukui and Y.~Kobayashi.
\newblock Second-order difference subspace.
\newblock In {\em Proc.\ of a computer vision / pattern recognition workshop}.
\newblock Higher-order extension of difference subspace.

\bibitem{Celik2009}
T.~Celik.
\newblock Unsupervised change detection in satellite images using principal
component analysis and $k$-means clustering.
\newblock {\em IEEE Geoscience and Remote Sensing Letters}, 6(4):772--776,
2009.

\bibitem{Daudt2018}
A.~Daudt, B.~Le~Saux, and A.~Boulch.
\newblock Fully convolutional siamese networks for change detection.
\newblock In {\em Proc.\ ICIP}, 2018.
\newblock Introduces CNN-based change detection and the OSCD dataset.

\bibitem{Ronneberger2015}
O.~Ronneberger, P.~Fischer, and T.~Brox.
\newblock U-Net: Convolutional networks for biomedical image segmentation.
\newblock In {\em Proc.\ MICCAI}, 2015.

\bibitem{He2016ResNet}
K.~He, X.~Zhang, S.~Ren, and J.~Sun.
\newblock Deep residual learning for image recognition.
\newblock In {\em Proc.\ CVPR}, 2016.

\bibitem{Gupta2019xBD}
R.~Gupta et~al.
\newblock xBD: A dataset for assessing building damage from satellite imagery.
\newblock In {\em Proc.\ CVPR Workshops}, 2019.

\end{thebibliography}

\end{document}